\chapter{Технологическая часть}

В данном разделе приведены средства реализации и листинг кода.

\section{Средства реализации}

В качестве языка программирования для реализации данной лабораторной работы был выбран высокопроизводительный язык C++ \cite{cpp17}, так как он предоставляет широкие возможности для эффективной реализации алгоритмов.

\section{Листинг кода}

В листингах \ref{lst:standard}, \ref{lst:kmp} и \ref{lst:bm} приведены листинги алгоритма сортировки пузырьком, вставками и выбором соответственно.

\begin{lstinputlisting}[
	caption={Стандартный алгоритм},
	label={lst:standard},
	style={cpp},
	linerange={6-26}
]{../string/search.cpp}
\end{lstinputlisting}

\begin{lstinputlisting}[
	caption={Алгоритм Кнута — Морриса — Пратта},
	label={lst:kmp},
	style={cpp},
	linerange={28-68}
]{../string/search.cpp}
\end{lstinputlisting}

\begin{lstinputlisting}[
	caption={Алгоритм Бойера — Мура},
	label={lst:bm},
	style={cpp},
	linerange={70-112}
]{../string/search.cpp}
\end{lstinputlisting}

\section{Тестирование функций}

В таблице~\ref{tbl:test} приведены тесты для функций, реализующих алгоритмы поиска подстроки в строке.
Тесты пройдены успешно.

\begin{table}[h!]
	\begin{center}
		\begin{tabular}{|c|c|c|}
			\hline
			\textbf{Строка1} & \textbf{Строка2} & \textbf{Ожидаемый результат} \\ \hline
			aaaaa      & a  & 1  \\ \hline
			aapo       & po & 2  \\ \hline
			poaa       & po & 0  \\ \hline
			qwerty     & a  & -1 \\ \hline
			popopopopo & po & 0  \\ \hline
		\end{tabular}
		\caption{Тестирование функций}
		\label{tbl:test}
	\end{center}
\end{table}

\section*{Вывод}

Правильный выбор инструментов разработки позволил эффективно реализовать алгоритмы, настроить модульное тестирование и выполнить исследовательский раздел лабораторной работы.
