\chapter*{Введение}
\addcontentsline{toc}{chapter}{Введение}

Поиск подстроки в строке — одна из простейших задач поиска информации.
Применяется в виде встроенной функции в текстовых редакторах, СУБД, поисковых машинах, языках программирования и т. п. \cite{Gusfield}

\section*{Задачи работы}

Цель лабораторной работы: изучить алгоритмы поиска подстроки в строке.

В рамках выполнения работы необходимо решить следующие задачи:
\begin{itemize}
	\item изучить стандартный алгоритм, алгоритмы Кнута — Морриса — Пратта и Бойера — Мура;
	\item реализовать данные алгоритмы;
	\item привести подробное описание работы каждого алгоритма.
\end{itemize}
