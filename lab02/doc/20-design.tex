\chapter{Конструкторская часть}

\section{Разработка алгоритмов}

\subsection{Стандартный алгоритм}

На рисунке \ref{img:product} приведена схема стандартного алгоритма умножения матриц.

\img{160mm}{product}{Cхема стандартного алгоритма умножения матриц}

Видно, что для стандартного алгоритма не существует лучшего и худшего случаев, как
таковых.


\section{Алгоритм Копперсмита — Винограда}

На рисунках \ref{img:winograd1} и \ref{img:winograd2} представлена схема алгоритма Копперсмита — Винограда.

\img{190mm}{winograd1}{Cхема алгоритма Копперсмита — Винограда}

\img{160mm}{winograd2}{Cхема алгоритма Копперсмита — Винограда}

Видно, что для алгоритма Виноградова худшим случаем являются матрицы нечетного размера, а лучшим четного, т.к. отпадает необходимость в последнем цикле.\\
В качестве оптимизаций можно:
\begin{itemize}
	\item заранее считать в MH и MV отрицательные произведения
	\item заменить выражения вида \code{a = a + ...} на \code{a += ...}
	\item в циклах по k сделать шаг 2, избавившись тем самым от двух операций умножения на каждую итерацию
\end{itemize}

\section*{Вывод}

На основе теоретических данных, полученных из аналитического раздела, были построены схемы обоих алгоритмов умножения матриц. Оценены лучшие и
худшие случаи их работы.
