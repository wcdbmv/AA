\chapter{Конструкторская часть}

\section{Разработка алгоритмов}

\subsection{Стандартный алгоритм}

На рисунке \ref{img:product} приведена схема стандартного алгоритма умножения матриц.

\img{160mm}{product}{Cхема стандартного алгоритма умножения матриц}

Видно, что для стандартного алгоритма не существует лучшего и худшего случаев, как таковых.


\section{Алгоритм Копперсмита — Винограда}

На рисунках \ref{img:winograd1} и \ref{img:winograd2} представлена схема алгоритма Копперсмита — Винограда.

\img{190mm}{winograd1}{Cхема алгоритма Копперсмита — Винограда}

\img{160mm}{winograd2}{Cхема алгоритма Копперсмита — Винограда (продолжение)}

Видно, что для алгоритма Виноградова худшим случаем являются матрицы нечетного размера, а лучшим четного, т.к. отпадает необходимость в последнем цикле.\\
В качестве оптимизаций можно:
\begin{itemize}
	\item заранее считать в MH и MV отрицательные произведения
	\item заменить выражения вида \code{a = a + ...} на \code{a += ...}
	\item в циклах по k сделать шаг 2, избавившись тем самым от двух операций умножения на каждую итерацию
\end{itemize}

\section{Трудоемкость алгоритмов}

\subsection{Стандартный алгоритм умножения матриц}

Трудоёмкость стандартного алгоритма умножения матриц
\begin{equation}
f = 2 + l(2 + 2 + n(2 + 2 + m(2 + 9)) = 11lmn + 4ln + 4l + 2.
\end{equation}

\subsection{Алгоритм Копперсмита — Винограда}

Трудоёмкость алгоритма Копперсмита — Винограда
\begin{enumerate}
	\item $f_{row} = 2 + l(2 + 2 + 0.5n(2 + 10)) = 6ln + 4l + 2$
	\item $f_{col} = 2 + n(2 + 2 + 0.5m(2 + 10)) = 6mn + 4n + 2$
	\item $f_{matrix} = 2 + l(2 + 2 + m(2 + 2 + 7 + 0.5n(2 + 22)) = 12lmn + 11lm + 4l + 2$
	\item $f_{end} = \begin{cases}
		2, & \text{чётная,}\\
		2 + 2 + l(2 + 2 + m(2 + 13)) = 15lm + 4l + 4, & \text{иначе.}
	\end{cases}$
\end{enumerate}

Итого, для худшего случая (нечетный размер матрицы) $f = f_{row} + f_{col} + f_{matrix} + f_{end} = 12lmn + 6ln + 6mn + 26lm + 12l + 4n + 10 \approx 12lmn$

Для лучшего случая (четный размер матрицы: $f = f_{row} + f_{col} + f_{matrix} + f_{end} = 12lmn + 6ln + 6mn + 11lm + 8l + 4n + 8 \approx 12lmn$

\subsection{Оптимизированный алгоритм Копперсмита — Винограда}

Трудоёмкость оптимизированного алгоритма Копперсмита — Винограда
\begin{enumerate}
	\item $f_{row} = 2 + l(2 + 2 + 0.5n(2 + 8)) = 5ln - l + 2$
	\item $f_{col} = 2 + n(2 + 2 + 0.5m(2 + 8)) = 5mn - n + 2$
	\item $f_{matrix} = 2 + l(2 + 2 + m(2 + 2 + 5 + 0.5n(2 + 15)) = 8.5lmn + 9lm + 4l + 2$
	\item $f_{end} = \begin{cases}
		2, & \text{чётная}\\
		2 + 2 + l(2 + 2 + m(2 + 10)) = 12lm + 4l + 4, & \text{иначе.}
	\end{cases}$
\end{enumerate}

Итого, для худшего случая (нечетный размер матрицы) $f = f_{row} + f_{col} + f_{matrix} + f_{end} = 8.5lmn + 5ln + 5mn + 21lm + 7l - n + 10 \approx 8lmn$

Для лучшего случая (четный размер матрицы): $f = f_{row} + f_{col} + f_{matrix} + f_{end} = 8.5lmn + 5ln + 5mn + 11lm + 3l - n + 8 \approx 8lmn$

\section*{Вывод}

На основе теоретических данных, полученных из аналитического раздела, были построены схемы обоих алгоритмов умножения матриц.
Оценены лучшие и худшие случаи их работы.
