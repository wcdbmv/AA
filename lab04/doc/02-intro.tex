\chapter*{Введение}
\addcontentsline{toc}{chapter}{Введение}

Многопоточность — способность центрального процессора (CPU) или одного ядра в многоядерном процессоре одновременно выполнять несколько процессов или потоков, соответствующим образом поддерживаемых операционной системой.
Этот подход отличается от многопроцессорности, так как многопоточность процессов и потоков совместно использует ресурсы одного или нескольких ядер: вычислительных блоков, кэш-памяти ЦПУ или буфера перевода с преобразованием (TLB).

В тех случаях, когда многопроцессорные системы включают в себя несколько полных блоков обработки, многопоточность направлена на максимизацию использования ресурсов одного ядра, используя параллелизм на уровне потоков, а также на уровне инструкций.
Поскольку эти два метода являются взаимодополняющими, их иногда объединяют в системах с несколькими многопоточными ЦП и в ЦП с несколькими многопоточными ядрами.

Многопоточная парадигма стала более популярной с конца 1990-х годов, поскольку усилия по дальнейшему использованию параллелизма на уровне инструкций застопорились.
Смысл многопоточности — квазимногозадачность на уровне одного исполняемого процесса.
Значит, все потоки процесса помимо общего адресного пространства имеют и общие дескрипторы файлов. Выполняющийся процесс имеет как минимум один (главный) поток.

Многопоточность (как доктрину программирования) не следует путать ни с многозадачностью, ни с многопроцессорностью, несмотря на то, что операционные системы, реализующие многозадачность, как правило, реализуют и многопоточность.

Достоинства:
\begin{itemize}
	\item облегчение программы посредством использования общего адресного пространства.
	\item меньшие затраты на создание потока в сравнении с процессами.
	\item повышение производительности процесса за счёт распараллеливания процессорных вычислений.
	\item если поток часто теряет кэш, другие потоки могут продолжать использовать неиспользованные вычислительные ресурсы.
\end{itemize}

Недостатки:
\begin{itemize}
	\item несколько потоков могут вмешиваться друг в друга при совместном использовании аппаратных ресурсов \cite{Nemirovsky};
	\item с программной точки зрения аппаратная поддержка многопоточности более трудоемка для программного обеспечения \cite{Olukotun};
	\item проблема планирования потоков;
	\item специфика использования. Вручную настроенные программы на ассемблере, использующие расширения MMX или AltiVec и выполняющие предварительные выборки данных, не страдают от потерь кэша или неиспользуемых вычислительных ресурсов. Таким образом, такие программы не выигрывают от аппаратной многопоточности и действительно могут видеть ухудшенную производительность из-за конкуренции за общие ресурсы \cite{intel}.
\end{itemize}
Несмотря на существующие недостатки, многопоточная парадигма имеет огромный потенциал, поэтому данная лабораторная работа будет посвящена распараллеливанию реализованного ранее алгоритма Винограда для умножения матриц.\\

\section*{Задачи работы}

В рамках выполнения работы необходимо решить следующие задачи:
\begin{itemize}
	\item изучить понятие параллельный вычислений;
	\item реализовать последовательный и параллельный алгоритм Винограда;
	\item сравнить временные характеристики реализованных алгоритмов экспериментально;
	\item на основании проделанной работы сделать выводы.
\end{itemize}

