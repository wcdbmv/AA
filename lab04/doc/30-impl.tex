\chapter{Технологическая часть}

В данном разделе приведены средства реализации и листинг кода.

\section{Средства реализации}

В качестве языка программирования для реализации данной лабораторной работы был выбран высокопроизводительный язык C++ \cite{cpp17}, так как он предоставляет широкие возможности для эффективной реализации алгоритмов.
Для распараллеливания алгоритма используется \code{std::thread} из стандартной библиотеки.

Для генерации псевдослучайных чисел использована функция из листинга \ref{lst:dont_try_to_guess}.

Для замера времени работы алгоритма использованы точнейшие функции библиотеки \code{std::chrono}

\section{Листинг кода}

В листингах \ref{lst:single-winograd} и \ref{lst:multi-winograd} приведены однопоточная и многопоточная реализации алгоритмов Копперсмита — Винограда.
Функция замера времени работы алгоритма приведена в листинге \ref{lst:count_time}.

\begin{lstinputlisting}[
	caption={Последовательный алгоритм Копперсмита — Винограда},
	label={lst:single-winograd},
	style={cpp},
	linerange={5-67}
]{../matrix/matrix.cpp}
\end{lstinputlisting}

\begin{lstinputlisting}[
	caption={Параллельный алгоритм Копперсмита — Винограда},
	label={lst:multi-winograd},
	style={cpp},
	linerange={69-167}
]{../matrix/matrix.cpp}
\end{lstinputlisting}

\begin{lstinputlisting}[
	caption={Продвинутый генератор псевдослучайных чисел},
	label={lst:dont_try_to_guess},
	style={cpp},
	linerange={8-12}
]{../performance_test.cpp}
\end{lstinputlisting}

\begin{lstinputlisting}[
	caption={Функция замера времени работы алгоритмов},
	label={lst:count_time},
	style={cpp},
	linerange={26-36}
]{../performance_test.cpp}
\end{lstinputlisting}

\section{Тестирование функций}

В таблице~\ref{tabular:test_rec} приведены тесты для функций, реализующих однопоточный и многопоточный алгоритмы Копперсмита — Винограда. Тесты пройдены успешно.

\begin{table}[h!]
	\begin{center}
		\begin{tabular}{c@{\hspace{7mm}}c@{\hspace{7mm}}c@{\hspace{7mm}}c@{\hspace{7mm}}c@{\hspace{7mm}}c@{\hspace{7mm}}}
			\hline
			Матрица 1 & Матрица 2 &Ожидаемый результат \\ \hline
			\vspace{4mm}
			$\begin{pmatrix}
			1 & 2 & 3\\
			1 & 2 & 3\\
			1 & 2 & 3
			\end{pmatrix}$ &
			$\begin{pmatrix}
			1 & 2 & 3\\
			1 & 2 & 3\\
			1 & 2 & 3
			\end{pmatrix}$ &
			$\begin{pmatrix}
			6 & 12 & 18\\
			6 & 12 & 18\\
			6 & 12 & 18
			\end{pmatrix}$ \\
			\vspace{2mm}
			\vspace{2mm}
			$\begin{pmatrix}
			1 & 2\\
			1 & 2
			\end{pmatrix}$ &
			$\begin{pmatrix}
			1 & 2\\
			1 & 2
			\end{pmatrix}$ &
			$\begin{pmatrix}
			3 & 6\\
			3 & 6
			\end{pmatrix}$ \\
			\vspace{2mm}
			\vspace{2mm}
			$\begin{pmatrix}
			2
			\end{pmatrix}$ &
			$\begin{pmatrix}
			2
			\end{pmatrix}$ &
			$\begin{pmatrix}
			4
			\end{pmatrix}$ \\
			\vspace{2mm}
			\vspace{2mm}
			$\begin{pmatrix}
			1 & -2 & 3\\
			1 & 2 & 3\\
			1 & 2 & 3
			\end{pmatrix}$ &
			$\begin{pmatrix}
			-1 & 2 & 3\\
			1 & 2 & 3\\
			1 & 2 & 3
			\end{pmatrix}$ &
			$\begin{pmatrix}
			0 & 4 & 6\\
			4 & 12 & 18\\
			4 & 12 & 18
			\end{pmatrix}$\\
			\vspace{2mm}
			\vspace{2mm}
			$\begin{pmatrix}
			1 & 2
			\end{pmatrix}$ &
			$\begin{pmatrix}
			1 & 2
			\end{pmatrix}$ &
			Не могут быть перемножены\\
		\end{tabular}
	\end{center}
	\caption{\label{tabular:test_rec} Тестирование функций}
\end{table}

\section*{Вывод}

Правильный выбор инструментов разработки позволил эффективно реализовать алгоритмы, настроить модульное тестирование и выполнить исследовательский раздел лабораторной работы.
