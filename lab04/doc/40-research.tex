\chapter{Исследовательская часть}

\section{Технические характеристики}

\begin{itemize}
	\item Операционная система: Ubuntu 19.10 64-bit.
	\item Память: 3,8 GiB.
	\item Процессор: Intel® Core™ i3-6006U CPU @ 2.00GHz c 2 физическими и 4 логическими ядрами
\end{itemize}

\section{Время выполнения алгоритмов}

Алгоритмы тестировались c помощью функции замера процессорного времени \code{std::chrono::high\_resolution\_clock::now()}.
Чтобы уменьшить случайные отклонения в измерениях, время считалось среднее для 5 запусков функций.

Результаты замеров приведены в таблицах \ref{tbl:runtime_even} и \ref{tbl:runtime_odd}.
На рисунках \ref{plt:runtime_even} и \ref{plt:runtime_odd} приведены графики зависимостей времени работы алгоритмов от размеров матриц на различном количестве потоков.

\begin{table}[h!]
	\begin{center}
		\begin{tabular}{|c|c|c|c|c|c|c|c|}
			\hline
			                 & \multicolumn{7}{c|}{\bfseries Время при кол-ве потоков, мс}                                        \\ \cline{2-8}
			\bfseries Размер & \bfseries 1 & \bfseries 2 & \bfseries 4 & \bfseries 8 & \bfseries 16 & \bfseries 32 & \bfseries 64
			\csvreader{inc/csv/runtime-even.csv}{}
			{\\\hline \csvcoli&\csvcolii&\csvcoliii&\csvcoliv&\csvcolv&\csvcolvi&\csvcolvii&\csvcolviii}
			\\\hline
		\end{tabular}
	\end{center}
	\caption{Время работы алгоритмов при чётных размерах матриц}
	\label{tbl:runtime_even}
\end{table}

\begin{table}
	\begin{center}
		\begin{tabular}{|c|c|c|c|c|c|c|c|}
			\hline
			                 & \multicolumn{7}{c|}{\bfseries Время при кол-ве потоков, мс}                                        \\ \cline{2-8}
			\bfseries Размер & \bfseries 1 & \bfseries 2 & \bfseries 4 & \bfseries 8 & \bfseries 16 & \bfseries 32 & \bfseries 64
			\csvreader{inc/csv/runtime-odd.csv}{}
			{\\\hline \csvcoli&\csvcolii&\csvcoliii&\csvcoliv&\csvcolv&\csvcolvi&\csvcolvii&\csvcolviii}
			\\\hline
		\end{tabular}
	\end{center}
	\caption{Время работы алгоритмов при нечётных размерах матриц}
	\label{tbl:runtime_odd}
\end{table}

\begin{figure}
	\centering
	\begin{tikzpicture}[scale=1.5]
		\begin{axis}[
			axis lines=left,
			xlabel=Размер,
			ylabel={Время, мс},
			legend pos=north west,
			ymajorgrids=true
		]
			\addplot table[x=size,y=thread1,col sep=comma]{inc/csv/runtime-even.csv};
			\addplot table[x=size,y=thread2,col sep=comma]{inc/csv/runtime-even.csv};
			\addplot table[x=size,y=thread4,col sep=comma]{inc/csv/runtime-even.csv};
			\addplot table[x=size,y=thread8,col sep=comma]{inc/csv/runtime-even.csv};
			\addplot table[x=size,y=thread16,col sep=comma]{inc/csv/runtime-even.csv};
			\addplot table[x=size,y=thread32,col sep=comma]{inc/csv/runtime-even.csv};
			\addplot table[x=size,y=thread64,col sep=comma]{inc/csv/runtime-even.csv};
			\legend{1 поток, 2 потока, 4 потока, 8 потоков, 16 потоков, 32 потока, 64 потока}
		\end{axis}
	\end{tikzpicture}
	\captionsetup{justification=centering}
	\caption{Зависимость времени работы алгоритма от размера квадратной матрицы (чётного)}
	\label{plt:runtime_even}
\end{figure}

\begin{figure}
	\centering
	\begin{tikzpicture}[scale=1.5]
		\begin{axis}[
			axis lines=left,
			xlabel=Размер,
			ylabel={Время, мс},
			legend pos=north west,
			ymajorgrids=true
		]
			\addplot table[x=size,y=thread1,col sep=comma]{inc/csv/runtime-odd.csv};
			\addplot table[x=size,y=thread2,col sep=comma]{inc/csv/runtime-odd.csv};
			\addplot table[x=size,y=thread4,col sep=comma]{inc/csv/runtime-odd.csv};
			\addplot table[x=size,y=thread8,col sep=comma]{inc/csv/runtime-odd.csv};
			\addplot table[x=size,y=thread16,col sep=comma]{inc/csv/runtime-odd.csv};
			\addplot table[x=size,y=thread32,col sep=comma]{inc/csv/runtime-odd.csv};
			\addplot table[x=size,y=thread64,col sep=comma]{inc/csv/runtime-odd.csv};
			\legend{1 поток, 2 потока, 4 потока, 8 потоков, 16 потоков, 32 потока, 64 потока}
		\end{axis}
	\end{tikzpicture}
	\captionsetup{justification=centering}
	\caption{Зависимость времени работы алгоритма от размера квадратной матрицы (нечётного)}
	\label{plt:runtime_odd}
\end{figure}

\section*{Вывод}

Наилучшее время алгоритм показал на 4 потоках, соответствующих количеству логических ядер, уменьшив время однопоточной реализации в $2,3$ раза.
Большее количество потоков в итоге немногим замедляет время необходимостью переключения контекста.
