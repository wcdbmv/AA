\chapter{Конструкторская часть}

\section{Разработка алгоритмов}

На рисунках \ref{img:selection-sort}, \ref{img:insertion-sort} и \ref{img:bubble-sort} представлены схемы алгоритмов сортировки выбором, вставками и пузырьком соответственно.

\imgext{height=70mm}{selection-sort}{pdf}{Схема алгоритма сортировки выбором}

\imgext{height=100mm}{insertion-sort}{pdf}{Схема алгоритма сортировки вставками}

\imgext{height=130mm}{bubble-sort}{pdf}{Схема алгоритма сортировки пузырьком}

\section{Трудоёмкость алгоритмов}

Пусть $n$ — длина массива (расстояние между \code{first} и \code{last}) и оператор сравнения в точности есть оператор меньше.
Лучший и худший случай для данных алгоритмов одинаковы и равны соответственно случаям отсортированного и обратно отсортированного массивов.

Подробно рассчитаем трудоёмкость алгоритма сортировки выбором.
В цикле \code{first} делается $n$ сравнений \code{first < last} и $n$ инкрементов переменной \code{first}.
В теле цикла \code{first} выполняется
\begin{equation}
	\sum_{k=1}^{n}{2k}=n(n+1)
\end{equation}
разыменовываний указателя и $n$ вызовов процедуры \code{swap}, которая выполняет 3 присваивания каждый раз.

Функция \code{min\_elem} возвращает указатель на минимальный элемент диапазона, длина которого постепенно уменьшается с $n$ до $1$.
Следовательно, суммарное количество сравнений в этой функции $n(n+1)$, а кол-во присваиваний в лучшем случае $n$ ($n$ раз минимальному значению присваивается первый элемент), в худшем — 
\begin{equation}
	\sum_{k=1}^{n}{k+1}=\frac12n(n+3).
\end{equation}
Кол-во разыменовываний указателя — $n(n+1)$.

Итого, трудоёмкость алгоритма сортировки выбором в лучшем случае:
\begin{equation}
	n+n+n(n+1)+3n+n(n+1)+n+n(n+1)=3n^2+9n=O(n^2),
\end{equation}
в худшем:
\begin{equation}
	n+n+n(n+1)+3n+n(n+1)+\frac12n(n+3)+n(n+1)=\frac{7n^2}{2}+\frac{17n}{2}=O(n^2).
\end{equation}

Рассчитывая аналогично, трудоёмкость алгоритма сортировки пузырьком в лучшем и худшем случае — $O(n^2)$, сортировки вставками в лучшем случае — $O(n)$, в худшем — $O(n^2)$.

\section*{Вывод}

Были разработаны схемы всех трех алгоритмов сортировки.
Для каждого из них были выделены и оценены лучшие и худшие случаи.
