\chapter{Технологическая часть}

В данном разделе приведены средства реализации и листинг кода.

\section{Средства реализации}

В качестве языка программирования для реализации данной лабораторной работы был выбран высокопроизводительный язык C++ \cite{cpp17}, так как он предоставляет широкие возможности для эффективной реализации алгоритмов.

Для генерации псевдослучайных чисел использована функция из листинга \ref{lst:dont_try_to_guess}.

Для замера времени работы алгоритма использованы точнейшие функции библиотеки \code{std::chrono}.

\section{Листинг кода}

В листингах \ref{lst:bubble-sort}, \ref{lst:insertion-sort} и \ref{lst:selection-sort} приведены листинги алгоритма сортировки пузырьком, вставками и выбором соответственно.
Функция замера времени работы алгоритма приведена в листинге \ref{lst:count_time}.

\begin{lstinputlisting}[
	caption={Алгоритм сортировки пузырьком},
	label={lst:bubble-sort},
	style={cpp},
	linerange={9-18}
]{../sort/sort.hpp}
\end{lstinputlisting}

\begin{lstinputlisting}[
	caption={Алгоритм сортировки вставками},
	label={lst:insertion-sort},
	style={cpp},
	linerange={20-27}
]{../sort/sort.hpp}
\end{lstinputlisting}

\begin{lstinputlisting}[
	caption={Алгоритм сортировки выбором},
	label={lst:selection-sort},
	style={cpp},
	linerange={29-34}
]{../sort/sort.hpp}
\end{lstinputlisting}

\begin{lstinputlisting}[
	caption={Продвинутый генератор псевдослучайных чисел},
	label={lst:dont_try_to_guess},
	style={cpp},
	linerange={7-11}
]{../performance_test.cpp}
\end{lstinputlisting}

\begin{lstinputlisting}[
	caption={Функция замера времени работы алгоритмов},
	label={lst:count_time},
	style={cpp},
	linerange={39-54}
]{../performance_test.cpp}
\end{lstinputlisting}

\section{Тестирование функций}

В таблице~\ref{tbl:test} приведены тесты для функций, реализующих алгоритмы сортировки. Тесты пройдены успешно.

\begin{table}[h!]
	\begin{center}
		\begin{tabular}{|c|c|c|}
			\hline
			Входной массив & Результат & Ожидаемый результат \\ 
			\hline
			$[1,2,3,4]$ & $[1,2,3,4]$  & $[1,2,3,4]$\\
			$[3,2,1]$  & $[1,2,3]$ & $[1,2,3]$\\
			$[5,6,2,4,-2]$  & $[-2,2,4,5,6]$  & $[-2,2,4,5,6]$\\
			$[4]$  & $[4]$  & $[4]$\\
			$[]$  & $[]$  & $[]$\\
			\hline
		\end{tabular}
		\caption{\label{tbl:test}Тестирование функций}
	\end{center}
\end{table}

\section*{Вывод}

Правильный выбор инструментов разработки позволил эффективно реализовать алгоритмы, настроить модульное тестирование и выполнить исследовательский раздел лабораторной работы.
