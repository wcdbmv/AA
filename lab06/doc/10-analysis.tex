\chapter{Аналитическая часть}

\section{Описание алгоритмов}

\subsubsection{Алгоритм полного перебора}

Алгоритм полного перебора для решения задачи коммивояжера предполагает рассмотрение всех возможных путей в графе и выбор наименьшего из них.

Такой подход гарантирует точное решение задачи, однако, так как задача относится к числу трансвычислительных, то уже при небольшом числе городов решение за приемлемое время невозможно.

\subsubsection{Муравьиный алгоритм}

Идея алгоритма основана на принципе работы колонии муравьев \cite{Bonabeau}. Колония муравьев рассматривается как многоагентная система, в которой каждый агент (муравей) функционирует автономно по очень простым правилам. В противовес почти примитивному поведению агентов, поведение всей системы получается разумным.

Каждый муравей определяет для себя маршрут, который необходимо пройти на основе феромона, который он ощущает, во время прохождения, каждый муравей оставляет феромон на своем пути, чтобы остальные муравьи могли по нему ориентироваться. В результате при прохождении каждым муравьем различного маршрута наибольшее число феромона остается на оптимальном пути. 

Самоорганизация колонии является результатом взаимодействия следующих компонентов:
\begin{itemize}
	\item случайность — муравьи имеют случайную природу движения;
	\item многократность — колония допускает число муравьев, достигающее от нескольких десятков до миллионов особей;
	\item положительная обратная связь — во время движения муравей откладывает феромон, позволяющий другим особям определить для себя оптимальный маршрут;
	\item отрицательная обратная связь — по истечении определенного времени феромон испаряется;
	\item целевая функция.
\end{itemize}

Пусть муравей обладает следующими характеристиками:
\begin{itemize}
	\item зрение — определяет длину ребра;
	\item обоняние — чувствует феромон;
	\item память — запоминает маршрут, который прошел.
\end{itemize}

Введем целевую функцию $\eta_{ij} = 1 / D_{ij}$, где $D_{ij}$ — расстояние из текущего пункта $i$ до заданного пункта $j$.

Посчитаем вероятности перехода в заданную точку по формуле \eqref{possibility}:
\begin{equation}
	\label{possibility}
	P_{kij} = \begin{cases}
		\frac{t_{ij}^a\eta_{ij}^b}{\sum_{q=1}^m t^a_{iq}\eta^b_{iq}}, \textrm{вершина не была посещена ранее муравьем k,} \\
		0, \textrm{иначе}
	\end{cases}
\end{equation}
где $a, b$ -- настраиваемые параметры, $t$ - концентрация феромона, причем $a + b = const$, а при $a = 0$ алгоритм вырождается в жадный \cite{Levitin}.

Когда все муравьи завершили движение происходит обновление феромона по формуле \eqref{pheromone1}:
\begin{equation}
	\label{pheromone1}
	t_{ij}(t+1) = (1-p)t_{ij}(t) + \Delta t_{ij}, \Delta t_{ij} = \sum_{k=1}^N t^k_{ij}
\end{equation}
где
\begin{equation}
	\label{pheromone2}
	\Delta t^k_{ij} = \begin{cases}
		Q/L_{k}, \textrm{ребро посещено k-ым муравьем,} \\
		0, \textrm{иначе}
	\end{cases}
\end{equation}
$L_{k}$ — длина пути k-ого муравья, $Q$ — настраивает концентрацию нанесения/испарения феромона, $N$ — количество муравьев.

\section*{Вывод}
По итогам аналитического раздела были описаны алгоритмы полного перебора и муравьиный алгоритм.
