\chapter*{Введение}
\addcontentsline{toc}{chapter}{Введение}

Муравьиный алгоритм — алгоритм для нахождения приближённых решений задач оптимизации на графах, таких, как задача коммивояжера, транспортная задача и аналогичных задач поиска маршрутов на графах.
Муравья нельзя назвать сообразительным.
Отдельный муравей не в состоянии принять ни малейшего решения.
Дело в том, что он устроен крайне примитивно: все его действия сводятся к элементарным реакциям на окружающую обстановку и своих собратьев.
Муравей не способен анализировать, делать выводы и искать решения.

Эти факты, однако, никак не согласуются с успешностью муравьев как вида.
Они существуют на планете более 100 миллионов лет, строят огромные жилища, обеспечивают их всем необходимым и даже ведут настоящие войны.
В сравнении с полной беспомощностью отдельных особей, достижения муравьев кажутся немыслимыми.

Добиться таких успехов муравьи способны благодаря своей социальности.
Они живут только в коллективах – колониях. Все муравьи колонии формируют так называемый роевой интеллект.
Особи, составляющие колонию, не должны быть умными: они должны лишь взаимодействовать по определенным – крайне простым – правилам, и тогда колония целиком будет эффективна.

В колонии нет доминирующих особей, нет начальников и подчиненных, нет лидеров, которые раздают указания и координируют действия.
Колония является полностью самоорганизующейся.
Каждый из муравьев обладает информацией только о локальной обстановке, не один из них не имеет представления обо всей ситуации в целом – только о том, что узнал сам или от своих сородичей, явно или неявно.
На неявных взаимодействиях муравьев, называемых стигмергией, основаны механизмы поиска кратчайшего пути от муравейника до источника пищи.

Каждый раз проходя от муравейника до пищи и обратно, муравьи оставляют за собой дорожку феромонов.
Другие муравьи, почувствовав такие следы на земле, будут инстинктивно устремляться к нему.
Поскольку эти муравьи тоже оставляют за собой дорожки феромонов, то чем больше муравьев проходит по определенному пути, тем более привлекательным он становится для их сородичей.
При этом, чем короче путь до источника пищи, тем меньше времени требуется муравьям на него – а следовательно, тем быстрее оставленные на нем следы становятся заметными.

В 1992 году в своей диссертации Марко Дориго (Marco Dorigo) предложил заимствовать описанный природный механизм для решения задач оптимизации \cite{Dorigo}.
Имитируя поведение колонии муравьев в природе, муравьиные алгоритмы используют многоагентные системы, агенты которых функционируют по крайне простым правилам.
Они крайне эффективны при решении сложных комбинаторных задач – таких, например, как задача коммивояжера \cite{TSPr}, первая из решенных с использованием данного типа алгоритмов \cite{Mueller}.

\section*{Задачи работы}

Цель лабораторной работы: изучить муравьиный алгоритм на материале решения задачи Коммивояжера.

В рамках выполнения работы необходимо решить следующие задачи:
\begin{itemize}
	\item дать постановку задачи;
	\item описать методы полного перебора и эвристический, основанный на муравьином алгоритме;
	\item реализовать данные методы;
	\item выбрать и подготовить классы данных;
	\item провести параметризацию метода, основанного на муравьином алгоритме;
	\item интерпретировать результаты и сравнить их с результатами метода полного перебора.
\end{itemize}
