\chapter*{Заключение}
\addcontentsline{toc}{chapter}{Заключение}

Таким образом, в ходе лабораторной работы было сделано следующее:
\begin{itemize}
	\item дана постановку задачи;
	\item описаны методы полного перебора и эвристический, основанный на муравьином алгоритме;
	\item реализованы данные методы;
	\item выбраны и подготовлены классы данных;
	\item проведена параметризация метода, основанного на муравьином алгоритме;
\end{itemize}

Были также сделаны выводы на основе полученных данных.
Эвристический метод, основанный на муравьином алгоритме имеет преимущество перед методом полного перебора за счет того, что способен работать с данными достаточно большого объема, в то время как полный перебор сильно ограничен размером данных.
Также были подобраны параметры для оптимальной работы метода на двух классах данных.
Однако, в отличии от полного перебора, эвристический алгоритм не гарантирует точность найденного им пути, есть вероятность, что путь будет не оптимален.
