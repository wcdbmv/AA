\chapter{Технологическая часть}

В данном разделе приведены средства реализации и листинг кода.

\section{Требования к ПО}

Программа должна обрабатывать матрицу смежностей методами полного перебора и эвристическим, основанным на муравьином алгоритме, подбирать параметры, на основе которых производятся оптимальные вычисления, выводить заданную матрицу смежностей, результат работы полного перебора и муравьиного алгоритма.

\section{Средства реализации}

В качестве языка программирования для реализации данной лабораторной работы был выбран высокопроизводительный язык C++ \cite{cpp17}, так как он предоставляет широкие возможности для эффективной реализации алгоритмов.

\section{Листинг кода}

В листингах \ref{lst:exhaustive} и \ref{lst:colony} представлены алгоритм полного перебора и класса колонии соответственно.

\begin{lstinputlisting}[
	caption={Алгоритм полного перебора},
	label={lst:exhaustive},
	style={cpp},
	linerange={16-45}
]{../algs/exhaustive_search.cpp}
\end{lstinputlisting}

\begin{lstinputlisting}[
	caption={Класс муравьиной колонии},
	label={lst:colony},
	style={cpp},
	linerange={7-41}
]{../algs/colony.hpp}
\end{lstinputlisting}

\section*{Вывод}

Правильный выбор инструментов разработки позволил эффективно реализовать алгоритмы, настроить модульное тестирование и выполнить исследовательский раздел лабораторной работы.
