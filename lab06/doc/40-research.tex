\chapter{Исследовательская часть}

\section{Технические характеристики}

\begin{itemize}
	\item Операционная система: Ubuntu 19.10 64-bit.
	\item Память: 3,8 GiB.
	\item Процессор: Intel® Core™ i3-6006U CPU @ 2.00GHz
\end{itemize}

\section{Постановка эксперимента}

В муравьином алгоритме вычисления производятся на основе настраиваемых параметров.
Рассмотрим два класса данных и подберем к ним параметры, при которых метод даст точный результат при минимальном количестве итераций.

Будем рассматривать матрицы размерности $10\times10$, так как иначе получение точного результата алгоритмом полного перебора слишком велико.

В качестве первого класса данных выделим матрицу смежностей, в которой все значения незначительно отличаются друг от друга, например, в диапазоне $[0, 10]$.
Вторым классом будут матрицы, где значения могут значительно отличаться, например $[1, 15000]$.

Будем запускать муравьиный алгоритм для всех значений $\alpha, P\in[0, 1]$, с шагом $= 0.1$, пока не будет найдено точное значение для каждого набора.
Если будет превышено допустимое количество итераций работа алгоритма при данных параметров будет завершена.

В результате тестирования будет выведена таблица со значениями $\alpha, \beta, P$, $iters$, $ dist$, где $iters$ — число итераций, за которое алгоритм нашел оптимальный путь, $dist$ — длина найденного пути, а $\alpha, \beta, p$ — настроечные параметры.

Ниже будут представлены результаты работы алгоритма для двух классов данных.

\subsection{Класс данных 1}

Матрица смежности для класса данных 1:
\begin{equation*}
	G = \begin{pmatrix}
		0,      &3      &10     &9      &8      &2      &1      &7      &2      &10     \\
		3,      &0      &8      &10     &3      &9      &6      &8      &4      &2      \\
		10,     &8      &0      &10     &3      &2      &7      &3      &10     &7      \\
		9,      &10     &10     &0      &3      &2      &2      &5      &10     &4      \\
		8,      &3      &3      &3      &0      &3      &6      &8      &5      &1      \\
		2,      &9      &2      &2      &3      &0      &10     &10     &2      &9      \\
		1,      &6      &7      &2      &6      &10     &0      &10     &4      &8      \\
		7,      &8      &3      &5      &8      &10     &10     &0      &6      &2      \\
		2,      &4      &10     &10     &5      &2      &4      &6      &0      &10     \\
		10,     &2      &7      &4      &1      &9      &8      &2      &10     &0
	\end{pmatrix}
\end{equation*}

В таблице ~\ref{T:log111} приведены результаты параметризаций метода решения задачи коммивояжера на основании муравьиного алгортима. Полный перебор определил оптимальную длину пути 22.

\begin{table}
	\caption{Таблица коэффициентов для класса данных №1}
	\begin{minipage}[h!]{0.10\hsize}\centering
		\begin{center}\resizebox{4\textwidth}{!}{%
			\begin{tabular}{c@{\hspace{5mm}}c@{\hspace{5mm}}c@{\hspace{5mm}}c@{\hspace{5mm}}c@{\hspace{5mm}}c}
				\toprule
				a        &b      &p      &iters &длина пути \\
				\midrule
				0       &1      &0      &50    &22\\
				0       &1      &0.1    &50    &22\\
				0       &1      &0.2    &50    &22\\
				0       &1      &0.3    &50    &22\\
				0       &1      &0.4    &50    &22\\
				0       &1      &0.5    &50    &22\\
				0       &1      &0.6    &50    &22\\
				0       &1      &0.7    &50    &22\\
				0       &1      &0.8    &50    &22\\
				0       &1      &0.9    &50    &22\\
				0       &1      &1      &7     &22\\
				\midrule
				0.1     &0.9    &0      &7     &22\\
				0.1     &0.9    &0.1    &24    &22\\
				0.1     &0.9    &0.2    &174   &22\\
				0.1     &0.9    &0.3    &174   &22\\
				0.1     &0.9    &0.4    &24    &23\\
				0.1     &0.9    &0.5    &24    &22\\
				0.1     &0.9    &0.6    &24    &22\\
				0.1     &0.9    &0.7    &174   &22\\
				0.1     &0.9    &0.8    &174   &22\\
				0.1     &0.9    &0.9    &119   &23\\
				0.1     &0.9    &1      &22    &23\\
				\midrule
				0.2     &0.8    &0      &24    &22\\
				0.2     &0.8    &0.1    &24    &23\\
				0.2     &0.8    &0.2    &24    &22\\
				0.2     &0.8    &0.3    &24    &22\\
				0.2     &0.8    &0.4    &174   &22\\
				0.2     &0.8    &0.5    &174   &22\\
				0.2     &0.8    &0.6    &24    &22\\
				0.2     &0.8    &0.7    &24    &22\\
				0.2     &0.8    &0.8    &24    &22\\
				0.2     &0.8    &0.9    &30    &22\\
				0.2     &0.8    &1      &2     &22\\
				\midrule
				0.3     &0.7    &0      &20    &22\\
				0.3     &0.7    &0.1    &20    &22\\
				0.3     &0.7    &0.2    &20    &22\\
				0.3     &0.7    &0.3    &20    &22\\
				0.3     &0.7    &0.4    &20    &22\\
				0.3     &0.7    &0.5    &20    &22\\
				0.3     &0.7    &0.6    &20    &23\\
				0.3     &0.7    &0.7    &20    &23\\
				0.3     &0.7    &0.8    &20    &23\\
				0.3     &0.7    &0.9    &102   &22\\
				0.3     &0.7    &1      &34    &22\\
				\bottomrule
			\end{tabular}}
			\label{T:log111}
		\end{center}
	\end{minipage}
	\hfill
	\begin{minipage}[!h]{0.50\hsize}\centering
		\begin{center}\resizebox{0.8\textwidth}{!}{%
			%\caption{Лог работы программы.}
			\begin{tabular}{c@{\hspace{5mm}}c@{\hspace{5mm}}c@{\hspace{5mm}}c@{\hspace{5mm}}c@{\hspace{5mm}}c}
				\toprule
				a        &b      &p      &iters &длина пути \\
				\midrule
				0.4     &0.6    &0      &20    &22\\
				0.4     &0.6    &0.1    &20    &22\\
				0.4     &0.6    &0.2    &20    &22\\
				0.4     &0.6    &0.3    &20    &22\\
				0.4     &0.6    &0.4    &20    &22\\
				0.4     &0.6    &0.5    &20    &22\\
				0.4     &0.6    &0.6    &20    &22\\
				0.4     &0.6    &0.7    &49    &22\\
				0.4     &0.6    &0.8    &20    &22\\
				0.4     &0.6    &0.9    &20    &22\\
				0.4     &0.6    &1      &8     &22\\
				\midrule
				0.5     &0.5    &0      &20    &22\\
				0.5     &0.5    &0.1    &20    &22\\
				0.5     &0.5    &0.2    &20    &22\\
				0.5     &0.5    &0.3    &20    &22\\
				0.5     &0.5    &0.4    &20    &22\\
				0.5     &0.5    &0.5    &20    &22\\
				0.5     &0.5    &0.6    &20    &22\\
				0.5     &0.5    &0.7    &20    &22\\
				0.5     &0.5    &0.8    &85    &22\\
				0.5     &0.5    &0.9    &163   &22\\
				0.5     &0.5    &1      &26    &22\\
				\midrule
				0.6     &0.4    &0      &20    &22\\
				0.6     &0.4    &0.1    &20    &23\\
				0.6     &0.4    &0.2    &20    &22\\
				0.6     &0.4    &0.3    &20    &22\\
				0.6     &0.4    &0.4    &20    &22\\
				0.6     &0.4    &0.5    &20    &22\\
				0.6     &0.4    &0.6    &20    &22\\
				0.6     &0.4    &0.7    &124   &23\\
				0.6     &0.4    &0.8    &49    &23\\
				0.6     &0.4    &0.9    &14    &22\\
				0.6     &0.4    &1      &65    &22\\
				\midrule
				0.7     &0.3    &0      &20    &22\\
				0.7     &0.3    &0.1    &20    &22\\
				0.7     &0.3    &0.2    &20    &22\\
				0.7     &0.3    &0.3    &20    &22\\
				0.7     &0.3    &0.4    &20    &22\\
				0.7     &0.3    &0.5    &20    &22\\
				0.7     &0.3    &0.6    &20    &22\\
				0.7     &0.3    &0.7    &20    &22\\
				0.7     &0.3    &0.8    &13    &22\\
				0.7     &0.3    &0.9    &16    &22\\
				0.7     &0.3    &1      &7     &22\\
				\bottomrule
			\end{tabular}}
			%\label{T:log}
		\end{center}
	\end{minipage}
\end{table}
\begin{table}[!h]
	\begin{center}
		\begin{tabular}{c@{\hspace{7mm}}c@{\hspace{7mm}}c@{\hspace{7mm}}c@{\hspace{7mm}}c@{\hspace{7mm}}c}
			\toprule
			a        &b      &p      &iters &длина пути \\
			\midrule
			0.4     &0.6    &0      &20    &22\\
			0.4     &0.6    &0.1    &20    &22\\
			0.4     &0.6    &0.2    &20    &22\\
			0.4     &0.6    &0.3    &20    &22\\
			0.4     &0.6    &0.4    &20    &22\\
			0.4     &0.6    &0.5    &20    &22\\
			0.4     &0.6    &0.6    &20    &22\\
			0.4     &0.6    &0.7    &49    &22\\
			0.4     &0.6    &0.8    &20    &22\\
			0.4     &0.6    &0.9    &20    &22\\
			0.4     &0.6    &1      &8     &22\\
			\midrule
			0.5     &0.5    &0      &20    &22\\
			0.5     &0.5    &0.1    &20    &22\\
			0.5     &0.5    &0.2    &20    &22\\
			0.5     &0.5    &0.3    &20    &22\\
			0.5     &0.5    &0.4    &20    &22\\
			0.5     &0.5    &0.5    &20    &22\\
			0.5     &0.5    &0.6    &20    &22\\
			0.5     &0.5    &0.7    &20    &22\\
			0.5     &0.5    &0.8    &85    &22\\
			0.5     &0.5    &0.9    &163   &22\\
			0.5     &0.5    &1      &26    &22\\
			\midrule
			0.6     &0.4    &0      &20    &22\\
			0.6     &0.4    &0.1    &20    &22\\
			0.6     &0.4    &0.2    &20    &22\\
			0.6     &0.4    &0.3    &20    &22\\
			0.6     &0.4    &0.4    &20    &22\\
			0.6     &0.4    &0.5    &20    &22\\
			0.6     &0.4    &0.6    &20    &22\\
			0.6     &0.4    &0.7    &124   &22\\
			0.6     &0.4    &0.8    &49    &22\\
			0.6     &0.4    &0.9    &14    &22\\
			0.6     &0.4    &1      &65    &22\\
			\bottomrule
		\end{tabular}
	\end{center}
\end{table}

\subsection{Класс данных 2}

Матрица смежности для класса данных 2:
\begin{equation*}
	G = \begin{pmatrix}
		0,      &13220  &5777   &10272  &2509   &12737  &11202  &13053  &2014   &3140   \\
		13220,  &0      &9305   &8955   &3974   &12863  &4135   &509    &13839  &2603   \\
		5777,   &9305   &0      &10978  &5521   &9191   &13678  &3453   &6444   &13320  \\
		10272,  &8955   &10978  &0      &13342  &10270  &8814   &14032  &1896   &6665   \\
		2509,   &3974   &5521   &13342  &0      &6897   &3215   &1483   &11523  &6752   \\
		12737,  &12863  &9191   &10270  &6897   &0      &9091   &5338   &9966   &6815   \\
		11202,  &4135   &13678  &8814   &3215   &9091   &0      &3973   &6879   &10087  \\
		13053,  &509    &3453   &14032  &1483   &5338   &3973   &0      &5463   &8252   \\
		2014,   &13839  &6444   &1896   &11523  &9966   &6879   &5463   &0      &4997   \\
		3140,   &2603   &13320  &6665   &6752   &6815   &10087  &8252   &4997   &0      \\
	\end{pmatrix}
\end{equation*}

В таблице \ref{T:log222} приведены результаты параметризаций метода решения задачи коммивояжера на основании муравьиного алгортима.
Полный перебор определил оптимальную длину пути 22.

\begin{table}[!h]
	\caption{Таблица коэффициентов для класса данных №2}
	\begin{center}\resizebox{0.575\textwidth}{!}{%
		\begin{tabular}{c@{\hspace{7mm}}c@{\hspace{7mm}}c@{\hspace{7mm}}c@{\hspace{7mm}}c@{\hspace{7mm}}c}
			\toprule
			a        &b      &p      &iters &длина пути \\
			\midrule
			0       &1      &0      &12    &40402\\
			0       &1      &0.1    &12    &40402\\
			0       &1      &0.2    &12    &40402\\
			0       &1      &0.3    &62    &40402\\
			0       &1      &0.4    &62    &40402\\
			0       &1      &0.5    &62    &40402\\
			0       &1      &0.6    &62    &40402\\
			0       &1      &0.7    &62    &40402\\
			0       &1      &0.8    &62    &40402\\
			0       &1      &0.9    &62    &40402\\
			0       &1      &1      &62    &40402\\
			\midrule
			0.1     &0.9    &0      &62    &40402\\
			0.1     &0.9    &0.1    &62    &40402\\
			0.1     &0.9    &0.2    &62    &40402\\
			0.1     &0.9    &0.3    &62    &40402\\
			0.1     &0.9    &0.4    &62    &40402\\
			0.1     &0.9    &0.5    &62    &40402\\
			0.1     &0.9    &0.6    &62    &40402\\
			0.1     &0.9    &0.7    &62    &40402\\
			0.1     &0.9    &0.8    &62    &40402\\
			0.1     &0.9    &0.9    &62    &40402\\
			0.1     &0.9    &1      &41    &40402\\
			\midrule
			0.2     &0.8    &0      &62    &40402\\
			0.2     &0.8    &0.1    &62    &40402\\
			0.2     &0.8    &0.2    &62    &40402\\
			0.2     &0.8    &0.3    &62    &40402\\
			0.2     &0.8    &0.4    &62    &40402\\
			0.2     &0.8    &0.5    &92    &40402\\
			0.2     &0.8    &0.6    &62    &40402\\
			0.2     &0.8    &0.7    &62    &40402\\
			0.2     &0.8    &0.8    &10    &40402\\
			0.2     &0.8    &0.9    &10    &40402\\
			0.2     &0.8    &1      &10    &40402\\
			\bottomrule
		\end{tabular}}
		\label{T:log222}
	\end{center}
\end{table}
\begin{table}
	\begin{minipage}[!h]{0.10\hsize}\centering
		\begin{center}\resizebox{4\textwidth}{!}{%
			\begin{tabular}{c@{\hspace{5mm}}c@{\hspace{5mm}}c@{\hspace{5mm}}c@{\hspace{5mm}}c@{\hspace{5mm}}c}
				\toprule
				a        &b      &p      &iters &длина пути \\
				\midrule
				0.3     &0.7    &0      &28    &40402\\
				0.3     &0.7    &0.1    &28    &40402\\
				0.3     &0.7    &0.2    &206   &40402\\
				0.3     &0.7    &0.3    &102   &40402\\
				0.3     &0.7    &0.4    &28    &40402\\
				0.3     &0.7    &0.5    &10    &40402\\
				0.3     &0.7    &0.6    &10    &40402\\
				0.3     &0.7    &0.7    &10    &40402\\
				0.3     &0.7    &0.8    &10    &40402\\
				0.3     &0.7    &0.9    &10    &40402\\
				0.3     &0.7    &1      &10    &40402\\
				\midrule
				0.4     &0.6    &0      &28    &40402\\
				0.4     &0.6    &0.1    &28    &40402\\
				0.4     &0.6    &0.2    &206   &40402\\
				0.4     &0.6    &0.3    &206   &40402\\
				0.4     &0.6    &0.4    &28    &40402\\
				0.4     &0.6    &0.5    &28    &40402\\
				0.4     &0.6    &0.6    &10    &40402\\
				0.4     &0.6    &0.7    &10    &40402\\
				0.4     &0.6    &0.8    &10    &40402\\
				0.4     &0.6    &0.9    &10    &40402\\
				0.4     &0.6    &1      &10    &40402\\
				\midrule
				0.5     &0.5    &0      &28    &40402\\
				0.5     &0.5    &0.1    &28    &40402\\
				0.5     &0.5    &0.2    &28    &40402\\
				0.5     &0.5    &0.3    &28    &40402\\
				0.5     &0.5    &0.4    &54    &40402\\
				0.5     &0.5    &0.5    &10    &40402\\
				0.5     &0.5    &0.6    &10    &40402\\
				0.5     &0.5    &0.7    &10    &40402\\
				0.5     &0.5    &0.8    &10    &40402\\
				0.5     &0.5    &0.9    &10    &40402\\
				0.5     &0.5    &1      &10    &40402\\
				\midrule
				0.6     &0.4    &0      &28    &40402\\
				0.6     &0.4    &0.1    &28    &40402\\
				0.6     &0.4    &0.2    &28    &40402\\
				0.6     &0.4    &0.3    &54    &40402\\
				0.6     &0.4    &0.4    &127   &40402\\
				0.6     &0.4    &0.5    &89    &40402\\
				0.6     &0.4    &0.6    &26    &40402\\
				0.6     &0.4    &0.7    &150   &40402\\
				0.6     &0.4    &0.8    &139   &40402\\
				0.6     &0.4    &0.9    &193   &40402\\
				0.6     &0.4    &1      &63    &40402\\
				\bottomrule
			\end{tabular}}
		\end{center}
	\end{minipage}
	\hfill
	\begin{minipage}[!h]{0.50\hsize}\centering
		\begin{center}\resizebox{0.8\textwidth}{!}{%
			\begin{tabular}{c@{\hspace{5mm}}c@{\hspace{5mm}}c@{\hspace{5mm}}c@{\hspace{5mm}}c@{\hspace{5mm}}c}
				\toprule
				a        &b      &p      &iters &длина пути \\
				\midrule
				0.7     &0.3    &0      &193   &40402\\
				0.7     &0.3    &0.1    &240   &40402\\
				0.7     &0.3    &0.2    &192   &40402\\
				0.7     &0.3    &0.3    &193   &40402\\
				0.7     &0.3    &0.4    &193   &40402\\
				0.7     &0.3    &0.5    &35    &40402\\
				0.7     &0.3    &0.6    &35    &40402\\
				0.7     &0.3    &0.7    &60    &40402\\
				0.7     &0.3    &0.8    &35    &40402\\
				0.7     &0.3    &0.9    &80    &40402\\
				0.7     &0.3    &1      &35    &40402\\
				\midrule
				0.8     &0.2    &0      &35    &40402\\
				0.8     &0.2    &0.1    &58    &40402\\
				0.8     &0.2    &0.2    &35    &40402\\
				0.8     &0.2    &0.3    &60    &40402\\
				0.8     &0.2    &0.4    &60    &40402\\
				0.8     &0.2    &0.5    &58    &40402\\
				0.8     &0.2    &0.6    &60    &40402\\
				0.8     &0.2    &0.7    &58    &40402\\
				0.8     &0.2    &0.8    &35    &40402\\
				0.8     &0.2    &0.9    &58    &40402\\
				0.8     &0.2    &1      &96    &40402\\
				\midrule
				0.9     &0.1    &0      &58    &40402\\
				0.9     &0.1    &0.1    &35    &40402\\
				0.9     &0.1    &0.2    &58    &40402\\
				0.9     &0.1    &0.3    &60    &40402\\
				0.9     &0.1    &0.4    &80    &40402\\
				0.9     &0.1    &0.5    &80    &40402\\
				0.9     &0.1    &0.6    &60    &40402\\
				0.9     &0.1    &0.7    &41    &40402\\
				0.9     &0.1    &0.8    &3     &40402\\
				0.9     &0.1    &0.9    &3     &40402\\
				0.9     &0.1    &1      &3     &40402\\
				\midrule
				1       &0   &0      &41    &40402\\
				1       &0   &0.1    &41    &40402\\
				1       &0   &0.2    &41    &40402\\
				1       &0   &0.3    &41    &40402\\
				1       &0   &0.4    &41    &40402\\
				1       &0   &0.5    &41    &40402\\
				1       &0   &0.6    &41    &40402\\
				1       &0   &0.7    &28    &40402\\
				1       &0   &0.8    &41    &40402\\
				1       &0   &0.9    &41    &40402\\
				1       &0   &1      &64    &40402\\
				\bottomrule
			\end{tabular}}
		\end{center}
	\end{minipage}
\end{table}

\section*{Вывод}

Таким образом, на основе полученных таблиц можно сделать вывод, что при классе данных, содержащем приблизительно равные значения наилучшими наборами стали ($\alpha = 0.2, \beta = 0.8, P = 1$), при данных значениях алгоритм нашел лучший путь за 2 запуска.
При наборах  ($\alpha = 0, \beta = 1, P = 1$), ($\alpha = 0.7, \beta = 0.3, P = 1$) алгоритм нашел путь за 7 итераций.

Для второго класса данных было определено, что при ($\alpha = 0.9, \beta = 0.1, P = 0.8$), ($\alpha = 0.9, \beta = 0.1, P = 0.9$), ($\alpha = 0.9, \beta = 0.1, P = 1$) алгоритм отработал за 3 итерации.
