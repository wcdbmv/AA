\chapter{Технологическая часть}

В данном разделе приведены средства реализации и листинг кода.

\section{Средства реализации}

В качестве языка программирования был выбран Go, так как он предоставляет широкие возможности и крайне удобный интерфейс для эффективной реализации асинхронной, параллельной обработки данных \cite{go}.

Для измерения времени использовалась стандартная библиотека \code{time}.
Так как основное время работы составляет ожидание \code{sleep}, то достаточно замерить время работы один раз.

\section{Листинг кода}

В листинге \ref{lst:conveyor} представлена реализация конвейерная обработка данных.

\begin{lstinputlisting}[
	caption={Алгоритм сортировки пузырьком},
	label={lst:conveyor},
	style={go}
]{../main.go}
\end{lstinputlisting}

\section{Тестирование функций}

Для тестирования были реализованы функции, представленные в листинге \ref{lst:test}.
Результаты тестирования представлены в таблице~\ref{tbl:test}. Видно, что тестирование пройдено успешно.

\begin{lstinputlisting}[
	caption={Тестовые задачи},
	label={lst:test},
	style={go},
	linerange={12-27}
]{../main_test.go}
\end{lstinputlisting}

\begin{table}[h!]
	\begin{center}
		\begin{tabular}{|c|c|c|}
			\hline
			Входные данные & Ожидаемый результат & Результат \\ 
			\hline
			1,3,4 & 24  & 24\\
			\hline
		\end{tabular}
		\caption{\label{tbl:test}Тестирование конвейерной обработки}
	\end{center}
\end{table}

\section*{Вывод}

Правильный выбор инструментов разработки позволил эффективно реализовать алгоритмы, настроить модульное тестирование и выполнить исследовательский раздел лабораторной работы.
