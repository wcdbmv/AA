\chapter*{Введение}
\addcontentsline{toc}{chapter}{Введение}

При обработке данных могут возникать ситуации, когда необходимо обработать множество данных последовательно несколькими алгоритмами. В этом случае удобно использовать конвейерную обработку данных. Это может быть полезно при следующих задачах:
\begin{itemize}
	\item шифровании данных;
	\item сортировки и фильтрации данных;
	\item и др.
\end{itemize}

Цель данной работы: получить навык организации асинхронного взаимодействия потоков на примере конвейерной обработки данных.

В рамках выполнения работы необходимо решить следующие задачи:
\begin{itemize}
	\item рассмотреть и изучить конвейерную обработку данных;
	\item реализовать конвейер с количеством лент не меньше трех в многопоточной среде;
	\item на основании проделанной работы сделать выводы.
\end{itemize}
