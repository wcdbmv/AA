\chapter{Технологическая часть}

В данном разделе приведены требования к программному обеспечению, средства реализации и листинги кода.

\section{Требования к ПО}

К программе предъявляется ряд требований:
\begin{itemize}
	\item на вход подаётся тип алгоритма (\code{[damerau\_]<matrix|recursive>}) и две строки;
	\item на выходе — искомое расстояние (для \code{matrix} — Левенштейна , для \code{damerau\_matrix}  или \code{damerau\_recursive} — Дамерау — Левенштейна);
	\item при указании ключа \code{-v|{-}-verbose} в матричных алгоритмах необходимо вывести матрицу;
	\item программа не должна аварийно завершаться при некорректном вводе пользователя.
\end{itemize}

\section{Средства реализации}

В качестве языка программирования для реализации данной лабораторной работы был выбран высокопроизводительный язык C++ \cite{cpp17}, предоставляющий широкие возможности для эффективной реализации алгоритмов.

\section{Листинг кода}

В листингах \ref{lst:levenshtein_core_detail_matrix}--\ref{lst:levenshtein_core_damerau_recursive} приведены реализации алгоритмов Левенштейна и Дамерау — Левенштейна.

\begin{lstinputlisting}[
	caption={Вспомогательные функции для матричных алгоритмов},
	label={lst:levenshtein_core_detail_matrix},
	style={cpp},
	linerange={6-44}
]{../levenshtein/core/levenshtein_core_detail_matrix.cpp}
\end{lstinputlisting}

\begin{lstinputlisting}[
	caption={Функция, реализующая матричный алгоритм Левенштейна},
	label={lst:levenshtein_core_matrix},
	style={cpp},
	linerange={5-23}
]{../levenshtein/core/levenshtein_core_matrix.cpp}
\end{lstinputlisting}

\begin{lstinputlisting}[
	caption={Функция, реализующая матричный алгоритм Дамерау — Левенштейна},
	label={lst:levenshtein_core_damerau_matrix},
	style={cpp},
	linerange={5-30}
]{../levenshtein/core/levenshtein_core_damerau_matrix.cpp}
\end{lstinputlisting}

\begin{lstinputlisting}[
	caption={Функция, реализующая рекурсивный алгоритм Дамерау — Левенштейна},
	label={lst:levenshtein_core_damerau_recursive},
	style={cpp},
	linerange={5-30}
]{../levenshtein/core/levenshtein_core_damerau_recursive.cpp}
\end{lstinputlisting}


В листинге \ref{lst:time} приведена реализация функции замера времени работы алгоритмов

\begin{lstinputlisting}[
	caption={Функция замера времени работы алгоритмов в тиках},
	label={lst:time},
	style={cpp},
	linerange={38-53}
]{../performance_test.cpp}
\end{lstinputlisting}

В таблице \ref{tabular:functional_test} приведен функциональные тесты для алгоритмов вычисления расстояния Левенштейна и Дамерау — Левенштейна. Все тесты пройдены успешно.

\begin{table}
	\begin{center}
		\begin{tabular}{|c|c|c|c|}
			\hline
			                    &                    & \multicolumn{2}{c|}{\bfseries Ожидаемый результат}    \\ \cline{3-4}
			\bfseries Строка 1  & \bfseries Строка 2 & \bfseries Левенштейн & \bfseries Дамерау — Левенштейн
			\csvreader{inc/csv/functional-test.csv}{}
			{\\\hline \csvcoli&\csvcolii&\csvcoliii&\csvcoliv}
			\\\hline
		\end{tabular}
		\caption{\label{tabular:functional_test} Функциональные тесты}
	\end{center}
\end{table}


\section*{Вывод}

Были разработаны и протестированы схемы трех алгоритмов: вычисления расстояния Левенштейна матрицей, вычисления расстояния Дамерау — Левенштейна матрицей и рекурсивно.
