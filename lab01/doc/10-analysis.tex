\chapter{Аналитическая часть}

Как говорилось ранее, расстояние Левенштейна между двумя строками — это минимальное количество операций вставки, удаления и замены, необходимых для превращения одной строки в другую.

Цены операций могут зависеть от вида операции (вставка, удаление, замена) и/или от участвующих в ней символов, отражая разную вероятность разных ошибок при вводе текста, и т. п. В общем случае:
\begin{itemize}
	\item $w(a,b)$ — цена замены символа $a$ на символ $b$
	\item $w(\lambda,b)$ — цена вставки символа $b$
	\item $w(a,\lambda)$ — цена удаления символа $a$
\end{itemize}
Расстояние Левенштейна является частным случаем этой задачи \cite{ifmo} при
\begin{itemize}
	\item $w(a,a)=0$
	\item $w(a,b)=1, \medspace a \neq b$
	\item $w(\lambda,b)=1$
	\item $w(a,\lambda)=1$
\end{itemize}

\section{Расстояние Левенштейна}

Пусть $S_{1}$ и $S_{2}$ — две строки (длиной $M$ и $N$ соответственно) над некоторым алфавитом, тогда расстояние Левенштейна $d(S_{1},S_{2})$ можно подсчитать по рекуррентной формуле $d(S_{1},S_{2}) = D(M,N)$, где\\
\[ 
D(i,j) = 
\left \{ \begin{aligned}
& 0, & i = 0, \medspace j = 0\\
& i, & i > 0, \medspace j = 0\\
& j, & i = 0, \medspace j > 0\\
& \min\{ &\\
& \qquad D(i, j - 1) + 1, & \\
& \qquad D(i - 1, j) + 1, & i > 0, \medspace j > 0\\
& \qquad D(i - 1, j - 1) + m(S_{1}[i], S_{2}[j]) &\\
& \} &
\end{aligned} \right. 
\]

\section{Расстояние Дамерау — Левенштейна}

Расстояние Дамерау — Левенштейна является модификацией алгоритма Левенштейна, отличается от него добавлением операции перестановки.

Расстояние Дамерау — Левенштейна $d(S_{1},S_{2})$ можно подсчитать по рекуррентной формуле $d(S_{1},S_{2}) = D(M,N)$, где\\
\[ 
D(i,j) = 
\left \{ \begin{aligned}
& 0, & i = 0, \medspace j = 0\\
& i, & i > 0, \medspace j = 0\\
& j, & i = 0,= \medspace j > 0\\
& \min\{ &\\
&   \qquad D(i, j - 1) + 1, &\\
&  \qquad D(i - 1, j) + 1, &i > 0, j > 0\\
&   \qquad D(i - 1, j - 1) + m(S_{1}[i], S_{2}[j]) &\\
& \},\\
& \min\{ &\\
&   \qquad D(i, j - 1) + 1, &\\
&   \qquad D(i - 1, j) + 1, &i, j > 1\\
&   \qquad D(i - 1, j - 1) + m(S_{1}[i], S_{2}[j]), &a_i = b_j - 1, a_i - 1 = b_j\\
&   \qquad D(i - 2, j - 2) + 1 &\\
& \} &
\end{aligned} \right.
\]

\section*{Вывод}

Формулы Левенштейна и Дамерау — Левенштейна для рассчета расстояния между строками задаются рекурсивно, а следовательно, алгоритмы могут быть реализованы рекурсивно или итерационно.
