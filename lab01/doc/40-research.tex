\chapter{Исследовательская часть}

\section{Пример работы}

Демонстрация работы программы приведена на рисунках \ref{img:demo_levenshtein_matrix}---\ref{img:demo_levenshtein_damerau}.

\boximg{59.5mm}{demo_levenshtein_matrix}{Демонстрация работы итеративного алгоритма Левенштейна}
\boximg{19.95mm}{demo_levenshtein_recursive}{Демонстрация работы рекурсивного алгоритма Левенштейна}
\boximg{58.2mm}{demo_levenshtein_damerau}{Демонстрация работы алгоритма Дамерау — Левенштейна}

\section{Технические характеристики}

\begin{itemize}
	\item Операционная система: Ubuntu 19.10 64-bit.
	\item Память: 3,8 GiB.
	\item Процессор: Intel® Core™ i3-6006U CPU @ 2.00GHz
\end{itemize}

\section{Время выполнения алгоритмов}

Алгоритмы тестировались c помощью функции замера процессорного времени \code{std::chrono::high\_resolution\_clock} при количестве повторов 100 для \hyperref[lst:levenshtein_matrix]{\code{matrix}}, \hyperref[lst:levenshtein_damerau_matrix]{\code{damerau\_matrix}}, и \hyperref[lst:levenshtein_damerau_recursive]{\code{damerau\_recursive}};

Результаты замеров приведены в таблицах \ref{tbl:performance-test0001-0010} и \ref{tbl:performance-test0010-1000}.
На рисунках \ref{plt:performance-damerau} и \ref{plt:performance-matrix} приведены графики зависимостей времени работы алгоритмов от длины строк.

\begin{table}[h!]
	\begin{center}
		\begin{tabular}{|c|c|c|c|}
			\hline
			                      & \multicolumn{3}{c|}{\bfseries Время, мс}                                    \\ \cline{2-4}
			\bfseries Длина строк & \bfseries matrix & \bfseries damerau\_matrix & \bfseries damerau\_recursive
			\csvreader{inc/csv/performance-test0001-0010.csv}{}
			{\\\hline \csvcoli&\csvcolii&\csvcoliii&\csvcoliv}
			\\\hline
		\end{tabular}
	\end{center}
	\caption{Замер времени для строк, размером до 10}
	\label{tbl:performance-test0001-0010}
\end{table}

\begin{table}
	\begin{center}
		\begin{tabular}{|c|c|c|}
			\hline
			                      & \multicolumn{2}{c|}{\bfseries Время, мс}     \\ \cline{2-3}
			\bfseries Длина строк & \bfseries matrix & \bfseries damerau\_matrix
			\csvreader{inc/csv/performance-test0010-1000.csv}{}
			{\\\hline \csvcoli&\csvcolii&\csvcoliii}
			\\\hline
		\end{tabular}
	\end{center}
	\caption{Замер времени для строк, размером до 1000}
	\label{tbl:performance-test0010-1000}
\end{table}

\begin{figure}
	\centering
	\begin{tikzpicture}
		\begin{axis}[
			axis lines=left,
			xlabel=Длина строк,
			ylabel={Время, мс},
			legend pos=north west,
			ymajorgrids=true
		]
			\addplot table[x=len,y=damerau_matrix,col sep=comma] {inc/csv/performance-test0001-0010.csv};
			\addplot table[x=len,y=damerau_recursive,col sep=comma] {inc/csv/performance-test0001-0010.csv};
			\legend{Матричный, Рекурсивный}
		\end{axis}
	\end{tikzpicture}
	\captionsetup{justification=centering}
	\caption{Зависимость времени работы алгоритма вычисления расстояния Дамерау — Левенштейна от длины строк (матричная и рекурсивная реализации)}
	\label{plt:performance-damerau}
\end{figure}

\begin{figure}
	\centering
	\begin{tikzpicture}
		\begin{axis}[
			axis lines=left,
			xlabel=Длина строк,
			ylabel={Время, мс},
			legend pos=north west,
			ymajorgrids=true
		]
			\addplot table[x=len,y=matrix,col sep=comma] {inc/csv/performance-test0100-1000.csv};
			\addplot table[x=len,y=damerau_matrix,col sep=comma] {inc/csv/performance-test0100-1000.csv};
			\legend{Левенштейн, Д. — Левенштейн}
		\end{axis}
	\end{tikzpicture}
	\captionsetup{justification=centering}
	\caption{Зависимость времени работы матричных реализаций алгоритмов Левенштейна и Дамерау — Левенштейна}
	\label{plt:performance-matrix}
\end{figure}

\section{Использование памяти}

Алгоритмы Левенштейна и Дамерау — Левенштейна не отличаются друг от друга с точки зрения использования памяти, следовательно, достаточно рассмотреть лишь разницу рекурсивной и матричной реализаций этих алгоритмов.

Максимальная глубина стека вызовов при рекурсивной реализации равна сумме длин входящих строк, соответственно, максимальный расход памяти (\ref{for:99})
\begin{equation}
(\mathcal{S}(src) + \mathcal{S}(dst)) \cdot (2 \cdot \mathcal{S}\mathrm{(std\!::\!string\&)} + 4 \cdot \mathcal{S}\mathrm{(size\_t)}),
\label{for:99}
\end{equation}
где $\mathcal{S}$ — оператор вычисления размера, $src$, $dst$ — строки, $\mathrm{size\_t}$ — целочисленный тип, $\mathrm{std\!::\!string\&}$ — тип указателя на строку.

Использование памяти при итеративной реализации теоретически равно
\begin{equation}
\mathcal{S}(src) \cdot \mathcal{S}(dst) \cdot \mathcal{S}\mathrm{(size\_t)} + 2\cdot(\mathcal{S}\mathrm{(std\!::\!string\&)} + \mathcal{S}\mathrm{(size\_t)}).
\end{equation}

Например, при вычислении расстояния Левеншейна для строк длиной 10, при размере типов $\mathrm{size\_t}$ и $\mathrm{std\!::\!string\&}$ 8 байтов получим:
\begin{itemize}
	\item для рекурсивной реализации $(10 + 10) \cdot (2\cdot8 + 4\cdot8) = 960$ байтов;
	\item для матричной реализации $10 \cdot 10 \cdot 8 + 2\cdot(8 + 8) = 832$ байта.
\end{itemize}

Можно заметить, что для корректной работы алгоритмов матрицу можно не хранить целиком, а использовать только текущую и предыдущую строки. При этом, если менять местами $src$ и $dst$ в случае $\mathcal{S}(src) > \mathcal{S}(dst)$, то тогда расход памяти составит
\begin{equation}
(2\min(\mathcal{S}(src), \mathcal{S}(dst)) + 1) \cdot \mathcal{S}\mathrm{(size\_t)} + 2\cdot\mathcal{S}\mathrm{(std\!::\!string\&)}.
\end{equation}
В примере выше для строк длиной 10: $(2(10 + 1)\cdot8 + 2\cdot8) = 192$ байта.

\section*{Вывод}

Рекурсивный алгоритм Левенштейна работает на порядок дольше итеративных реализаций, время его работы увеличивается в геометрической прогрессии. Алгоритм Дамерау — Левенштейна работает немногим дольше алгоритма Левенштейна, т. к. в нём добавлены дополнительные проверки.

Но по расходу памяти итеративные алгоритмы проигрывают рекурсивному: максимальный размер используемой памяти в них растёт как произведение длин строк, в то время как у рекурсивного алгоритма — как сумма длин строк. Однако в итеративных алгоритмах можно добиться ещё меньшего порядка роста, равному минимуму из длин строк, если не использовать всю матрицу целиком.
