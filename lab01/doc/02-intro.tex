\chapter*{Введение}
\addcontentsline{toc}{chapter}{Введение}

Расстояние Левенштейна между двумя строками в теории информации и компьютерной лингвистике --- это минимальное количество операций вставки одного символа, удаления одного символа и замены одного символа на другой, необходимых для превращения одной строки в другую.

Впервые задачу поставил в 1965 году советский математик Владимир Левенштейн \cite{Levenshtein} при изучении последовательностей 0--1, впоследствии более общую задачу для произвольного алфавита связали с его именем.

Расстояние Левенштейна и его обобщения активно применяются: для исправления ошибок в слове (в поисковых системах, базах данных, при вводе текста, при автоматическом распознавании отсканированного текста или речи); для сравнения текстовых файлов утилитой \code{diff} и ей подобными (здесь роль «символов» играют строки, а роль «строк» — файлы); в биоинформатике для сравнения генов, хромосом и белков \cite{Gasfield}.

Расстояние Дамерау — Левенштейна является модификацией расстояния Левенштейна: к операциям вставки, удаления и замены символов, определённых в расстоянии Левенштейна добавлена операция транспозиции (перестановки) символов. Дамерау показал, что 80\% человеческих ошибок при наборе текстов составляют перестановки соседних символов, пропуск символа, добавление нового символа, и ошибка в символе.

\section*{Задачи работы}

\begin{itemize}
	\item Изучение алгоритмов Левенштейна и Дамерау--Левенштейна.
	\item Применение методов динамического программирования для реализации алгоритмов.
	\item Получение практических навыков реализации алгоритмов Левенштейна и Дамерау — Левенштейна.
	\item Сравнительный анализ алгоритмов на основе экспериментальных данных.
	\item Подготовка отчета по лабораторной работе.
\end{itemize}
